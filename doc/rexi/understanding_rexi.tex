%% LyX 2.1.3 created this file.  For more info, see http://www.lyx.org/.
%% Do not edit unless you really know what you are doing.
\documentclass[english]{article}
\usepackage[T1]{fontenc}
\usepackage[latin9]{inputenc}
\usepackage{babel}
\begin{document}

\title{Understanding the rational approximation of the exponential integrator
(REXI)}


\author{Martin Schreiber <M.Schreiber@exeter.ac.uk>}

\maketitle
This document serves as the basis for implementing the rational approximation
of the exponential integrator (REXI). Here, we purely focus on the
linear part of the shallow-water equations (SWE) and show the different
steps to approximate solving this linear part with an exponential
integrator. This paper is mainly summarises previous work on REXI.


\section{Problem formulation}

We use the advective formulation of the SWE with a full linearization
with perturbation (see \cite{Formulations of the shallow-water equations})
with $U:=(h,u,v)^{T}$, yielding

\[
L(U):=\left(\begin{array}{ccc}
 & H\delta_{x} & H\delta_{y}\\
g\delta_{x} &  & -f\\
g\delta_{y} & f
\end{array}\right)U
\]


Here, we neglect all non-linear terms, assume only small perturbations
(hence negligible ones) around the average surface height and negligible
non-linear terms.

Then, the time evolution of the PDE with the subscript $t$ denoting
the derivative in time is given by

\[
U_{t}:=L(U).
\]


We continue writing the linear operator in matrix-style $L$ and applying
$L$ on $U$ as $L.U$.

\[
U_{t}:=L.U
\]


It is further worth noting, that this system describes an oscillatory
system, hence the operator $L$ has imaginary eigenvalues.


\section{Exponential integrator}

Linear systems of equations are well known to be solvable with exponential
integrators for arbitrary time step sizes via

\[
U(t):=e^{Lt}U(0).
\]


However, this is typically quite expensive to compute and analytic
solutions only exist for some simplified system of equations, see
e.g. \cite{Formulations of the shallow-water equations} for f-plane
shallow-water equations. These exponential integrators can be approximated
with rational functions and this paper is on giving insight into this
approximation.


\section{Underlying idea of rational approximation}

Terry et. al. developed a rational approximation of the exponential
integrator, see \cite{High-order time-parallel approximation of evolution operators}.
First, we like to get more insight into it with a one-dimensional
formulation before applying REXI to a rational approximation of a
linear operator. Our main target is to find an approximation of an
operator with an \emph{exponential shape}, such as $e^{ix}$, which
(in one-dimension) is given by a function $f(x)$.


\subsection{Step A) Approximation of solution space}

First, we assume that we can use Gaussian curves as basis functions
for our approximation and they are somehow naturally related to this
problem (Gaussians are given by exponential functions). So first,
we find an approximation of one of our underlying Gaussian basis function

\[
\psi_{s}(x):=(4\pi)^{-\frac{1}{2}}e^{-x^{2}/(4s^{2})}
\]


In this formulation, $s$ can be interpreted as the horizontal ``stretching''
of the basis function. Note the similarities to the Gaussian distribution,
but by dropping certain parts of the vertical scaling as it is required
for probability distributions. We can now approximate our function
$f(x)$ with the Gaussian distributions:

\[
f(x)\approx\sum_{m=-M}^{M}b_{m}\psi_{s}(x+ms)
\]


with $M$ controling the interval of approximation (\textasciitilde{}size
of ``domain of interest'') and $s$ the accuracy of integration
(\textasciitilde{}resolution in ``domain of interest'').


\subsection{Step B) Approximation of basis function}

The second step is the approximation of the basis function $\psi_{s}(x)$
itself via

\[
\psi_{s}(x)\approx2Re\left(\sum_{l=-L}^{L}\frac{a_{j}}{ix-(\mu+i\,l)}\right).
\]
See the paper for these mathematical magic tricks.... The weights
are derived via optimization in Fourier space.


\subsection{Step C) Approximation of the approximation}

We then combine the approximation (B) of the approximation (A), yielding

\[
f(x)\approx2Re\left(\sum_{n=-M-L}^{M+L}\frac{c_{n}}{ix-s(\mu+i\,n)}\right)
\]


which gives us the REXI for a one-dimensional function with precomputed
values of $c_{n}$ and $\mu$.


\section{REXI, our little dog}

In the following, we use $L:=\tau L'$ and assume an a-priori fixed
time step size, making a REXI approximation more efficient. Then,
the REXI approximation is given by

\begin{equation}
exp(\tau L')\approx\sum_{k=-K}^{K}\beta_{k}(L-\alpha_{k})^{-1}\label{eq:rexi}
\end{equation}


The coefficients $\alpha_{k}$ (corresponding to $s(\mu+i\,n)$ in
step C for the one-dimensional formulation) can be precomputed based
on a one-dimensional approximation, see the paper, and can be interpreted
as shifts of the rational approximations. The coefficients $\beta_{k}$
(corresponding to $c_{n}$ in step C) are describing the scaling of
the basis function and depend on the solution itself.

Note an important property (see Sec. 3.3 in \cite{High-order time-parallel approximation of evolution operators}).
There's an anti-symmetry in the $\alpha_{i}$ coefficients, which
avoids computing half of the inverses:

\[
\overline{(L-\alpha)^{-1}u_{0}}=(L-\overline{\alpha})^{-1}u_{0}
\]



\section{Computing inverse of $(L-\alpha)^{-1}$}

We still have to compute the inverse of $(L-\alpha)$ which can be
very expensive if directly computing it. Here, we consider a specialization
on the shallow-water equations given above with

\[
L(U):=\left(\begin{array}{ccc}
 & H\delta_{x} & H\delta_{y}\\
g\delta_{x} &  & -f\\
g\delta_{y} & f
\end{array}\right)U,
\]
\[
U_{t}:=L(U)
\]
and we set $g:=1$ and the average height $H:=1$. For the REXI method,
we now have to solve the systems of equations given by

\[
(L-\alpha).U=U0
\]
with $U_{0}$ the initial conditions. According to \cite{An invariant theory of the linearized shallow water equations with rotation and its application to a sphere and a plane},
instead of solving this relatively large system of equations we can
split the problem into an elliptic one for the height which then allows
to use an explicit formulation for the velocities. We use the abbreviation$\vec{v}:=(u,v)$
in the following paragraph. Using the formulation in \cite{High-order time-parallel approximation of evolution operators},
the height can be computed with the elliptic equation given by

\begin{equation}
(\nabla^{2}-(\alpha^{2}+f^{2}))h(\tau)=\frac{\alpha^{2}+f^{2}}{\alpha}(h(0)+H\nabla\cdot(A.v(0))\label{eq:elliptic_height}
\end{equation}
with

\[
A:=\frac{1}{\alpha^{2}+f^{2}}\left(\begin{array}{cc}
\alpha & -f\\
f & \alpha
\end{array}\right).
\]


We can rearrange this equation by using the abbreviations $\kappa:=\alpha^{2}+f^{2}$
and $\gamma:=\alpha^{-1}$ in the following way:

\[
(\nabla^{2}-\kappa)\,h(\tau)=\frac{\kappa}{\alpha}\left(h(0)+H\nabla\cdot(A.v(0)\right)
\]


\[
(\nabla^{2}-\kappa)\,h(\tau)=\frac{\kappa}{\alpha}h(0)+\frac{1}{\alpha}H\nabla\cdot\left(\begin{array}{cc}
\alpha & -f\\
f & \alpha
\end{array}\right).v(0)
\]


\[
(\nabla^{2}-\kappa)\,h(\tau)=\frac{\kappa}{\alpha}h(0)+\frac{1}{\alpha}H\left(\begin{array}{cc}
\alpha & -f\\
f & \alpha
\end{array}\right)\nabla\cdot v(0)
\]


\begin{equation}
(\nabla^{2}-\kappa)h(\tau)=\frac{\kappa}{\alpha}h(0)-\frac{Hf}{\alpha}\nabla\times v(0)+H\nabla\cdot v(0)
\end{equation}
Here, the $\alpha$ denote the only terms with imaginary numbers.
On the right hand side, we see an update-like scheme $h(0)$ in the
first scheme, then a vorticity-like formulation $\times$, and an
advection part $\nabla$ of the height. Once computed the height,
the velocities can be directly computed via

\[
\vec{v}(\tau)=-A.\vec{v}(0)+A.\nabla h
\]


\begin{equation}
\vec{v}(\tau)=-A.\vec{(v}(0)+\nabla h)\label{eq:elliptic_velocity}
\end{equation}
giving us our final solution

\[
U:=(h,v,u)^{T}.
\]
One final scaling has to be done: the exponential is computing $e^{\tau L}$,
hence the $\tau$ has to be included in the operator $L$. There are
basically two different ways: The first one is rescaling all parameters
by $\tau$:

\[
g':=\tau g
\]


\[
f':=\tau f
\]


\[
h_{0}':=\tau h_{0}
\]
The second way is to factor the $\tau$ parameter out:

\[
(\tau L-\alpha).U(\tau)=U0
\]


\[
(L-\frac{\alpha}{\tau}).U(\tau)\tau=U0
\]
So instead of solving for $U(\tau)$, we are solving for $U(\tau)\tau$
and have to divide the computed solution by $\tau$ in the end.


\section{Bringing everything together}

Using the spectral methods (e.g. in SWEET), we can directly solve
the height Eq. (\ref{eq:elliptic_height}) and then solver for the
velocity in Eq. (\ref{eq:elliptic_velocity}). Then, the problem is
reduced to computing the REXI as given in Eq. (\ref{eq:rexi}). We
like to note again, that the $\alpha_{i}$ are independent of the
system $L$ to solve, and only the $\beta_{i}$ depend on the system
$L$.


\section{Acknowledgements}

Thanks to Pedro \& Terry!
\begin{thebibliography}{1}
\bibitem{Formulations of the shallow-water equations}Formulations
of the shallow-water equations, M. Schreiber

\bibitem{High-order time-parallel approximation of evolution operators}High-order
time-parallel approximation of evolution operators, T. Haut et. al.

\bibitem{An asymptotic parallel-in-time method for highly oscillatory PDEs}An
asymptotic parallel-in-time method for highly oscillatory PDEs, T.
Haut et. al.

\bibitem{An invariant theory of the linearized shallow water equations with rotation and its application to a sphere and a plane}An
invariant theory of the linearized shallow water equations with rotation
and its application to a sphere and a plane, N. Paldor et. al.\end{thebibliography}

\end{document}
