%% LyX 2.1.4 created this file.  For more info, see http://www.lyx.org/.
%% Do not edit unless you really know what you are doing.
\documentclass[english]{article}
\usepackage[T1]{fontenc}
\usepackage[latin9]{inputenc}
\usepackage{amsmath}
\usepackage{esint}

\makeatletter

%%%%%%%%%%%%%%%%%%%%%%%%%%%%%% LyX specific LaTeX commands.
%% Because html converters don't know tabularnewline
\providecommand{\tabularnewline}{\\}

%%%%%%%%%%%%%%%%%%%%%%%%%%%%%% User specified LaTeX commands.
\usepackage{babel}

\makeatother

\usepackage{babel}
\begin{document}

\title{Understanding the Rational Approximation of the Exponential Integrator
(REXI)}


\author{Martin Schreiber <M.Schreiber@exeter.ac.uk>\\
 Pedro S. Peixoto <pedrosp@ime.usp.br>\\
 et al.}

\maketitle
This document serves as the basis for implementing the Rational approximation
of the EXponential Integrator (REXI). Here, we purely focus on the
linear part of the shallow-water equations (SWE) and show the different
steps to approximate solving this linear part with an exponential
integrator. This paper mainly summarises previous work on REXI.


\section{Problem formulation}

We use linearised shallow water equations (SWE) with respect to a
rest state with mean water depth of $\eta_{0}$ and defined for perturbations
of height $\eta$ (see \cite{Schreiber:Formulations of the shallow-water equations},
note that the sign is inverted). The linear operator ($L$) may be
written as

\[
L(U):=\left(\begin{array}{ccc}
0 & -\eta_{0}\partial_{x} & -\eta_{0}\partial_{y}\\
-g\partial_{x} & 0 & f\\
-g\partial_{y} & -f & 0
\end{array}\right)U
\]
where $U:=(\eta,u,v)^{T}$. Here, we neglect all non-linear terms
and consider $f$ constant (f-plane approximation). The time evolution
of the PDE, with the subscript $t$ denoting the derivative in time,
is given by

\[
U_{t}=L(U).
\]
It is further worth noting, that this system describes an oscillatory
system (2D wave equation), hence the operator $L$ is hyperbolic and
has imaginary eigenvalues. Linear initial value differential problems
are well known to be solvable with exponential integrators for arbitrary
time step sizes via 
\[
U(t)=e^{Lt}U(0).
\]
see e.g. \cite{Moler:Nineteen Dubious Ways to Compute the Exponential of a Matrix}.
However, this is typically quite expensive to compute and analytic
solutions only exist for some simplified system of equations, see
e.g. \cite{Schreiber:Formulations of the shallow-water equations}
for f-plane shallow-water equations. These exponential integrators
can be approximated with rational functions and this paper is on giving
insight into this approximation.


\section{1D rational approximation}

Terry et. al. \cite{Terry:High-order time-parallel approximation of evolution operators}
developed a rational approximation of the exponential integrator.
First, we like to get more insight into it with a one-dimensional
formulation before applying REXI to a rational approximation of a
linear operator. Our main target is to find an approximation of an
operator with a \emph{complex exponential shape}, in our case $e^{ix}$,
which (in one-dimension) is given as a function $f(x)$. We will end
up in an approximation given by the following rational approximation:

\[
e^{ix}\approx\sum_{n=-N}^{N}\frac{\beta_{n}}{ix-\alpha_{n}}
\]
with complex coefficients $\alpha_{n}$ and $\beta_{n}$. We point
out that the coefficients $\alpha_{n}$ will always have non zero
real part, so no singularity occurs with the rational function.


\subsection{Step A: Approximation of solution space}

First, we assume that we can use Gaussian curves as basis functions
for our approximation. So first we find an approximation of one of
our underlying Gaussian basis function

\[
\psi_{h}(x):=(4\pi)^{-\frac{1}{2}}e^{-x^{2}/(4h^{2})}
\]


In this formulation, $h$ can be interpreted as the horizontal ``stretching''
of the basis function. Note the similarities to the Gaussian distribution,
but by dropping certain parts of the vertical scaling as it is required
for probability distributions. We can now approximate our function
$f(x)$ with a superposition of basis functions $\psi_{h}(x)$ by

\[
f(x)\approx\sum_{m=-M}^{M}b_{m}\psi_{h}(x+mh)
\]
with $M$ controlling the interval of approximation (\textasciitilde{}size
of ``domain of interest'') and $h$ will be related to the accuracy
of integration (\textasciitilde{}resolution in ``domain of interest'').
We choose $h$ small enough so that the support of the Fourier transform
of $f$ is mainly localised within $[-1/(2h),1/(2h)]$, i.e. almost
zero outside this interval. $M$ is chosen such that the approximation
will be adequate in the interval $|x|<Mh$.

To compute the coefficients $b_{m}$, we rewrite the previous equation
in Fourier space with 
\[
\frac{\hat{f}(\xi)}{\hat{\psi_{h}}(\xi)}=\sum_{m=\infty}^{\infty}b_{m}e^{2\pi imh\xi},
\]
where the $\hat{\cdot}$ symbols indicate the Fourier transforms of
the respective functions. The $b_{m}$ are now the Fourier coefficients
of the series for the function $\frac{\hat{f}(\xi)}{\hat{\psi_{h}}(\xi)}$
and can be calculated as (see \cite{Terry:High-order time-parallel approximation of evolution operators},
page 11), 
\[
b_{m}=h\intop_{-\frac{1}{2h}}^{\frac{1}{2h}}e^{-2\pi imh\xi}\frac{\hat{f}(\xi)}{\hat{\psi_{h}}(\xi)}d\xi,
\]
for $m$ integer and $1/h$ defines the periodicity of the trigonometric
basis function.

Since we are interested in approximating $f(x)=e^{ix}$, we can simplify
the equation by using the response in frequency space $\hat{f}(\xi)=\delta(\xi-\frac{1}{2\pi})$,
where here $\delta$ is the Dirac distribution, and

\[
b_{m}=h\,e^{-imh}\hat{\psi_{h}}(\frac{1}{2\pi})^{-1}.
\]
The Fourier transform of the Gaussian function is well known and given
by

\begin{eqnarray*}
\hat{\psi_{h}}(\xi) & = & \intop_{-\infty}^{\infty}\frac{1}{\sqrt{4\pi}}e^{-\left(\frac{x}{2h}\right)^{2}}e^{-2\pi ix\xi}dx\\
 & = & \frac{1}{\sqrt{4\pi}}\intop_{-\infty}^{\infty}e^{-\left(\left(\frac{x}{2h}\right)^{2}+2\pi ix\xi+(2h\pi i\xi)^{2}-(2h\pi i\xi)^{2}\right)}dx\\
 & = & \frac{1}{\sqrt{4\pi}}e^{-(2h\pi\xi)^{2}}\intop_{-\infty}^{\infty}e^{-\left(\frac{x}{2h}+2h\pi i\xi\right)^{2}}dx\\
 & = & \frac{1}{\sqrt{4\pi}}e^{-(2h\pi\xi)^{2}}\intop_{-\infty}^{\infty}e^{-\left(\frac{x}{2h}\right)^{2}}dx\\
 & = & he^{-(2h\pi\xi)^{2}}
\end{eqnarray*}
where we used that $\intop_{-\infty}^{\infty}e^{-\left(\frac{x}{2h}\right)^{2}}dx=h\,\sqrt{4\pi}$
and completed squares in the exponential term. For the case $\xi=\frac{1}{2\pi}$,
we get

\[
\hat{\psi}_{h}\left(\frac{1}{2\pi}\right)=h\,e^{-h^{2}}.
\]
Finally, one can obtain the equation

\[
b_{m}=h\,e^{-imh}\frac{1}{h\,e^{-h^{2}}}=e^{-imh}e^{h^{2}}
\]
to compute the coefficients $b_{m}$ for $f(x)=e^{ix}$.


\subsection{Step B: Approximation of basis function}

The second step is the approximation of the basis function $\psi_{h}(x)$
itself with a rational approximation, see \cite{Damle:Near optimal rational approximations of large data sets}.
Our basis function is given by

\[
\psi_{h}(x):=(4\pi)^{-\frac{1}{2}}e^{-x^{2}/(4h^{2})}
\]
and a close-to-optimal approximation of $\psi_{1}(x)$ with a sum
of rational functions is given by

\[
\psi_{1}(x)\approx Re\left(\sum_{l=-L}^{L}\frac{a_{l}}{ix+(\mu+i\,l)}\right)
\]
with the $\mu$ and $a_{l}$ given in \cite{Damle:Near optimal rational approximations of large data sets},
Table 1. We can generalise this approximation to arbitrary chosen
$h$ via 
\[
\psi_{h}(x)\approx Re\left(\sum_{l=-L}^{L}\frac{a_{l}}{i\frac{x}{h}+(\mu+i\,l)}\right).
\]


The theory of how these coefficients are calculated are in \cite{Damle:Near optimal rational approximations of large data sets}
and will not be described here. Therefore, we assume that the coefficients
$a_{l}$ are given.


\subsection{Step C: Approximation of the approximation}

We then combine the approximation (B) into the approximation (A),
yielding the approximation $\tilde{f}$ for $f$ given by 
\[
\tilde{f}(x)=\sum_{m=-M}^{M}b_{m}\psi_{h}(x+mh)=\sum_{m=-M}^{M}b_{m}Re\left(\sum_{l=-L}^{L}\frac{a_{l}}{i\frac{x+mh}{h}+(\mu+i\,l)}\right)
\]


\[
=\sum_{m=-M}^{M}b_{m}\sum_{l=-L}^{L}Re\left(\frac{ha_{l}}{ix+h(\mu+i(m+l))}\right).
\]


We further like to simplify this equation and we observe, that for
$n:=m+l$, the denominator is equal. We can hence express parts of
the denomiator in terms of $n:=m+l$ by

\begin{equation}
\alpha_{n}:=s(\mu+in).\label{eq:alpha}
\end{equation}
Now, we merge the $b_{m}$ and $a_{l}$ coefficients and first have
a look at the $b_{m}$ which is complex valued. We observe the following
property: Assuming that we want to compute the real value of $f(x)$,
only the real value of $b_{m}$ has to be merged with the sum, since
the imaginary component would be dropped afterwards. This allows us
to move the $Re(b_{m})$ values inside the $\sum_{L}$:

\[
Re(f(x)):=Re\left(\sum_{m=-M}^{M}\,\,\sum_{l=-L}^{L}\frac{Re(b_{m})\,ha_{l}}{ix+h(\mu+i(m+l))}\right).
\]
Now we can collect all nominators with equivalent denominator (if
$n=m+l$ and by using $\delta$ as the Kronecker delta), yielding

\begin{equation}
\beta_{n}^{Re}:=h\sum_{m=-M\,}^{M}\sum_{l=-L}^{L}Re(b_{m})ha_{l}\delta(n,\,m+l)\label{eq:beta_re}
\end{equation}
for real values $f(x)$ and

\begin{equation}
\beta_{n}^{Im}:=h\sum_{m=-M\,}^{M}\sum_{l=-L}^{L}Im(b_{m})a_{l}\delta(n,\,m+l)\label{eq:beta_im}
\end{equation}
for complex values of $f(x)$. Note, that the complexity of this operation
is $O(N\,L)$, which is negligible for small $N$ and $L$. This can
be optimized by using $L_{1}=max(-L,n-M)$ and $L_{2}=min(L,n+M)$
(see \cite{Terry:High-order time-parallel approximation of evolution operators})
and we can compute $\beta_{n}^{Re}$ with

\begin{equation}
\beta_{n}^{Re}:=h\sum_{k=L_{1}}^{L_{2}}Re(b_{n-k})a_{k},\label{eq:beta_re_fast}
\end{equation}
and $\beta_{n}^{Im}$ correspondingly.

Finally, we get the REXI approximation

\[
e^{ix}\approx\sum_{n=-N}^{N}Re\left(\frac{\beta_{n}^{Re}}{ix+\alpha_{n}}\right)+i\,Re\left(\frac{\beta_{n}^{Im}}{ix+\alpha_{n}}\right).
\]



\subsection{Example coefficents}

For a better understanding and discussion of the poles, we provide
some examplanory coefficients for $\alpha_{n}$ and $\beta_{n}^{Re}$
computed with $M:=2$ and $h:=0.2$:

{\scriptsize{}}%
\begin{tabular}{|c|c|c|}
\hline 
$n$ & {\scriptsize{}$\alpha_{n}$} & {\scriptsize{}$\beta_{n}$}\tabularnewline
\hline 
\hline 
{\scriptsize{}-13} & {\scriptsize{}(-0.863064302175, -2.6)} & {\scriptsize{}(-2.0794560075645e-08,5.312368394177e-09)}\tabularnewline
\hline 
{\scriptsize{}-12} & {\scriptsize{}(-0.863064302175, -2.4)} & {\scriptsize{}(-1.8562925598646e-08,-1.6892470811809e-07)}\tabularnewline
\hline 
{\scriptsize{}-11} & {\scriptsize{}(-0.863064302175, -2.2)} & {\scriptsize{}(6.8570271350932e-07,-4.4377515257134e-08)}\tabularnewline
\hline 
{\scriptsize{}-10} & {\scriptsize{}(-0.863064302175, -2)} & {\scriptsize{}(1.9470768200785e-07,2.1186231739561e-06)}\tabularnewline
\hline 
{\scriptsize{}-9} & {\scriptsize{}(-0.863064302175, -1.8)} & {\scriptsize{}(3.037169144916e-06,-3.8007524015554e-06)}\tabularnewline
\hline 
{\scriptsize{}-8} & {\scriptsize{}(-0.863064302175, -1.6)} & {\scriptsize{}(-0.00020292956274934,-9.4793805592883e-05)}\tabularnewline
\hline 
{\scriptsize{}-7} & {\scriptsize{}(-0.863064302175, -1.4)} & {\scriptsize{}(0.00051562027155282,0.0033198141762956)}\tabularnewline
\hline 
{\scriptsize{}-6} & {\scriptsize{}(-0.863064302175, -1.2)} & {\scriptsize{}(0.023802856324805,-0.020097812439831)}\tabularnewline
\hline 
{\scriptsize{}-5} & {\scriptsize{}(-0.863064302175, -1)} & {\scriptsize{}(-0.16210306892042,-0.057527918763957)}\tabularnewline
\hline 
{\scriptsize{}-4} & {\scriptsize{}(-0.863064302175, -0.8)} & {\scriptsize{}(0.083936569694558,0.55379453117192)}\tabularnewline
\hline 
{\scriptsize{}-3} & {\scriptsize{}(-0.863064302175, -0.6)} & {\scriptsize{}(0.87683903065806,-0.58136186212318)}\tabularnewline
\hline 
{\scriptsize{}-2} & {\scriptsize{}(-0.863064302175, -0.4)} & {\scriptsize{}(-0.87618099667542,-0.6444132979014)}\tabularnewline
\hline 
{\scriptsize{}-1} & {\scriptsize{}(-0.863064302175, -0.2)} & {\scriptsize{}(-0.2112750856805,0.51693268636776)}\tabularnewline
\hline 
{\scriptsize{}0} & \textbf{\scriptsize{}(-0.863064302175, 0)} & \textbf{\scriptsize{}(0.21113064943379,1.1012434042446e-07)}\tabularnewline
\hline 
{\scriptsize{}1} & {\scriptsize{}(-0.863064302175, 0.2)} & {\scriptsize{}(-0.2112752777559,-0.51693263772868)}\tabularnewline
\hline 
{\scriptsize{}2} & {\scriptsize{}(-0.863064302175, 0.4)} & {\scriptsize{}(-0.87618105783081,0.6444131761443)}\tabularnewline
\hline 
{\scriptsize{}3} & {\scriptsize{}(-0.863064302175, 0.6)} & {\scriptsize{}(0.87683907406497,0.58136183517238)}\tabularnewline
\hline 
{\scriptsize{}4} & {\scriptsize{}(-0.863064302175, 0.8)} & {\scriptsize{}(0.083936534477108,-0.55379454106338)}\tabularnewline
\hline 
{\scriptsize{}5} & {\scriptsize{}(-0.863064302175, 1)} & {\scriptsize{}(-0.16210304313401,0.057527824955638)}\tabularnewline
\hline 
{\scriptsize{}6} & {\scriptsize{}(-0.863064302175, 1.2)} & {\scriptsize{}(0.023802980792584,0.020097827969804)}\tabularnewline
\hline 
{\scriptsize{}7} & {\scriptsize{}(-0.863064302175, 1.4)} & {\scriptsize{}(0.00051562077173168,-0.0033196934926057)}\tabularnewline
\hline 
{\scriptsize{}8} & {\scriptsize{}(-0.863064302175, 1.6)} & {\scriptsize{}(-0.0002030221163996,9.4802957184526e-05)}\tabularnewline
\hline 
{\scriptsize{}9} & {\scriptsize{}(-0.863064302175, 1.8)} & {\scriptsize{}(3.0281700967037e-06,3.7434363774526e-06)}\tabularnewline
\hline 
{\scriptsize{}10} & {\scriptsize{}(-0.863064302175, 2)} & {\scriptsize{}(2.2311216999616e-07,-2.1234907990132e-06)}\tabularnewline
\hline 
{\scriptsize{}11} & {\scriptsize{}(-0.863064302175, 2.2)} & {\scriptsize{}(6.871098037128e-07,5.5123982746463e-08)}\tabularnewline
\hline 
{\scriptsize{}12} & {\scriptsize{}(-0.863064302175, 2.4)} & {\scriptsize{}(-2.1322288893395e-08,1.6899352278552e-07)}\tabularnewline
\hline 
{\scriptsize{}13} & {\scriptsize{}(-0.863064302175, 2.6)} & {\scriptsize{}(-2.0738399377275e-08,-5.6642992624128e-09)}\tabularnewline
\hline 
\end{tabular}{\scriptsize \par}


\section{REXI on linear operators}

In this section, we investigate the linear operator $L$ with the
rational approximation.


\subsection{Reducing number of computations for $L$}

Note the property (see Sec. 3.3 in \cite{Terry:High-order time-parallel approximation of evolution operators})
for the $\alpha_{n}$ and $\beta_{n}$: There is an anti-symmetry
around the central pole with
\begin{equation}
\alpha_{-n}=\bar{\alpha}_{+n}\label{eq:alpha_conjugate_symmetry}
\end{equation}
and
\begin{equation}
\beta_{-n}=\bar{\beta}_{+n}\label{eq:beta_conjugate_symmetry}
\end{equation}
In particular, with $Im(\alpha_{0})=Im(\beta_{0})=0$, there is a
zero imaginary number for the central pole.

Furthermore, it holds that 
\[
\overline{(L+\alpha)^{-1}U(0)}=(L+\overline{\alpha})^{-1}U(0)
\]
with the overbar denoting the complex conjugate. We can then reformulate
the approximation
\begin{eqnarray*}
e^{\tau L} & \approx & \sum_{n=-N}^{N}Re\left(\frac{\beta_{n}^{Re}}{\tau L+\alpha_{n}}\right)\\
 & = & \sum_{n=-N}^{-1}Re\left(\frac{\beta_{n}^{Re}}{\tau L+\alpha_{n}}\right)+Re\left(\frac{\beta_{0}^{Re}}{\tau L+\alpha_{0}}\right)+\sum_{n=-N}^{-1}Re\left(\frac{\beta_{N+n+1}^{Re}}{\tau L+\alpha_{N+n+1}}\right)
\end{eqnarray*}
and using the properties \eqref{eq:alpha_conjugate_symmetry} and
\eqref{eq:beta_conjugate_symmetry}, we can write this as
\begin{align*}
\sum_{n=-N}^{-1}Re\left(\frac{\beta_{n}^{Re}}{\tau L+\alpha_{n}}\right)+Re\left(\frac{\beta_{0}^{Re}}{\tau L+\alpha_{0}}\right)+\sum_{n=-N}^{-1}Re\left(\frac{\overline{\beta_{N}^{Re}}}{\tau L+\overline{\alpha_{N}}}\right)\\
=\sum_{n=-N}^{-1}\left(Re\left(\frac{\beta_{n}^{Re}}{\tau L+\alpha_{n}}+\overline{\left(\frac{\beta_{n}^{Re}}{\tau L+\alpha_{n}}\right)}\right)\right)+Re\left(\frac{\beta_{0}^{Re}}{\tau L+\alpha_{0}}\right).
\end{align*}


\textbf{(TODO: Is this reformulation in the last line really allowed?!?
We are computing the conjugate on $\tau L$)}\\
Since the imaginary parts cancel out for $Re(a+\overline{a})=a+\overline{a}$
and with $\alpha_{0}$ and $\beta_{0}$ being only real-valued, we
can simplify the equation to
\[
e^{\tau L}\approx\sum_{n=-N}^{N}Re\left(\frac{\beta_{n}^{Re}}{\tau L+\alpha_{n}}\right)=\sum_{n=-N}^{-1}\left(2\frac{\beta_{n}^{Re}}{\tau L+\alpha_{n}}\right)+\frac{\beta_{0}^{Re}}{\tau L+\alpha_{0}}.
\]
This allows us to reduce the computational amount almost by a factor
of two for solving $(L+\alpha)^{-1}$ giving the real valued solution
\[
e^{\tau L}\approx\sum_{n=0}^{N}\frac{\gamma_{n}^{Re}}{\tau L+\alpha_{n}}
\]
with 
\[
\gamma_{n}:=\begin{cases}
\begin{array}{c}
\beta_{0}\\
2\beta_{n}
\end{array} & \begin{array}{c}
for\,n=0\\
else
\end{array}\end{cases}
\]



\subsection{Matrix exponential}

\label{sec:mat_exp} We would like to apply REXI to a formulation
such as 
\[
U(t)=e^{\tau L}U(0).
\]
To see the relationship between the approximation of $e^{ix}$ with
$e^{\tau L}$ we assume that $L$ is skew hermitian and therefore
has only purely imaginary eigenvalues, and maybe decomposed as $\Sigma\Lambda\Sigma^{H}$,
yielding 
\begin{equation}
e^{\tau L}=\sum_{k=0}^{\infty}\frac{(\tau L)^{k}}{k!}=\sum_{k=0}^{\infty}\frac{\tau\Sigma\Lambda^{k}\Sigma^{H}}{k!}=\Sigma\left(\sum_{k=0}^{\infty}\frac{(\tau\Lambda)^{k}}{k!}\right)\Sigma^{H}=\Sigma e^{\tau\Lambda}\Sigma^{H},\label{eq:expL}
\end{equation}
where we used the orthonormality of $\Sigma$ to cancel it out from
the summation, and

\[
e^{\tau\Lambda}=\left(\begin{array}{ccc}
...\\
 & e^{i\lambda_{j}\tau}\\
 &  & ...
\end{array}\right)
\]
where we have explicitly detached the imaginary unit from the eigenvalues,
therefore $\lambda_{n}$ are assumed real. Since $e^{\tau\Lambda}$
is diagonal, it can be eigenvalue-wise approximated in the same way
as in $e^{ix}$ with REXI.

Although $L$ has imaginary eigenvalues, we wish to evaluate the $e^{\tau L}U(0)$,
which is real valued. Therefore, we will use the real approximation
of $e^{ix}$ 
\begin{equation}
exp(\tau L)\approx Re\left(\sum_{n=-N}^{N}\beta_{n}(\tau L+\alpha_{n})^{-1}\right),\label{eq:rexi}
\end{equation}
where $\beta_{n}$ is given by equation \eqref{eq:beta_re} and $\alpha_{n}$
by equation \eqref{eq:alpha}.

Given the linear operator $L$ to be applied on $U(0)$, we continue
to show the relation of the linear operator to the REXI approximation
of the real term
\[
e^{\tau L}\approx\sum_{n=-N}^{N}Re\left(\frac{\beta_{n}^{Re}}{\tau L+\alpha_{n}}\right)
\]
as we use it later on. Next, we rewrite $L$ with our EV decomposition
as
\begin{eqnarray}
e^{\tau L} & \approx & \sum_{n=-N}^{N}Re\left(\beta_{n}^{Re}(\tau\Sigma\Lambda\Sigma^{H}+\alpha_{n}I)^{-1}\right)\nonumber \\
 & = & \sum_{n=-N}^{N}Re\left(\beta_{n}^{Re}(\Sigma(\tau\Lambda\Sigma^{H}+\alpha_{n}\Sigma^{H}))^{-1}\right)\nonumber \\
 & = & \sum_{n=-N}^{N}Re\left(\beta_{n}^{Re}(\Sigma(\tau\Lambda+\alpha_{n}\Sigma^{H}\Sigma)\Sigma^{H})^{-1}\right)\nonumber \\
 & = & \sum_{n=-N}^{N}Re\left(\beta_{n}^{Re}\Sigma^{H}(\tau\Lambda+\alpha_{n})^{-1}\Sigma\right)\nonumber \\
 & = & \Sigma^{H}\left[\sum_{n=-N}^{N}Re\left(\beta_{n}^{Re}(\tau\Lambda+\alpha_{n})^{-1}\right)\right]\Sigma\nonumber \\
 & = & \Sigma^{H}e^{\tau\Lambda}\Sigma\label{eq:expLeigen}
\end{eqnarray}


\textbf{@ALL: equation \eqref{eq:expL} and \eqref{eq:expLeigen}
should match, but are not. }

Note, that we use the same REXI approximation for different $\lambda_{j}$.
Hence, the approximation has to be sufficiently accurate over the
entire range of all Eigenvalues$\lambda_{j}$ which is what we use
the approximation for.


\subsection{Choosing $h$ and $M$}

Some important points about the choice of $M$ and $h$ have to be
made at this point. We know that $e^{ix}$ is accurately approximated
with REXI for the interval $|x|<hM$, where $h$ is chosen small enough
to obtain a good approximation in step (A), and $M$ will define the
interval size and number of approximation points. In the matrix case,
$M$ has to be chosen so that $hM>t\bar{\lambda}$, where $\bar{\lambda}=\max_{n}|\lambda_{n}|$,
in order to capture all wavelengths of $L$. In other other words,
$hM$ need to be set to capture the fastest wave. Note that if this
is used as a time stepping method, with time step $t=\tau$, then,
the larger the timestep, the larger $M$ will be. Exact evaluations
of the choices for $h$ and $M$ may be done based on equation (3.6)
of \cite{Terry:High-order time-parallel approximation of evolution operators}.

TODO: We have to investigate this by far in more depth. Can we possibly
use a standard time step restriction of the linear equations?


\subsection{Handling $\tau$ in REXI\label{sub:Handling-tau-in-REXI}}

We reformulate the REXI approximation scheme given by

\[
(\tau L+\alpha)^{-1}U(\tau)=U(0)
\]
and by factoring $\tau$ out, yielding

\[
(L+\frac{\alpha}{\tau})^{-1}U(\tau)\tau^{-1}=U(0)
\]
So instead of solving for $U(\tau)$, we are solving for $U^{\tau}(\tau):=U(\tau)\tau^{-1}$
as well as $\alpha^{\tau}:=\frac{\alpha}{\tau}$.

To summarize, we have to solve the system of equations given by

\begin{equation}
(L+\alpha^{\tau})^{-1}U^{\tau}(\tau)=U(0)\label{eq:unit_rexi_timestep}
\end{equation}
with $U(0)$ the initial conditions. For sake of simplicity, we stick
to the formulation without the $\tau$ notation.


\subsection{Computing inverse of $(L+\alpha)^{-1}$}

For computing the inverse, arbitrary solvers can be used. However
we like to note, that $\alpha$ is a complex number. Hence, requiring
solvers with support for solving in complex space.

We wish to solve the differential problem for each time step 
\[
(-L+\alpha)U=U(0)
\]
so that $U=(-L+\alpha)^{-1}U_{0}$. Note, the change in sign before
$L$ for convenience which has to be accounted for afterwards. We
will do this converting the problem into and elliptic equation.\textbf{}\\


First, lets expand the equations with the definition of $L$, 
\begin{eqnarray}
fv+g\eta_{x}+\alpha u & = & u_{0},\label{eq:mom_u}\\
fu+g\eta_{y}+\alpha v & = & v_{0},\label{eq:mom_v}\\
\bar{\eta}(u_{x}+v_{y})+\alpha\eta & = & \eta_{0}.\label{eq:mass}
\end{eqnarray}


Let $f$ be constant (f-plane approximation), $\delta:=u_{x}+v_{y}$
be the wind divergence, $\zeta:=v_{x}-u_{y}$ be the wind (relative)
vorticity and $\Delta\eta:=\eta_{xx}+\eta_{yy}$ the Laplacian of
the fluid depth. We will re-write the problem in a divergence-vorticity
formulation by taking 2 steps. First, sum the $\partial_{x}$ of equation
\eqref{eq:mom_u} and the $\partial_{y}$ of equation \eqref{eq:mom_v},
yielding 
\begin{equation}
-f\zeta+g\Delta\eta+\alpha\delta=\delta_{0}.\label{eq:zeta}
\end{equation}
Then subtract the $\partial_{y}$ of equation \eqref{eq:mom_u} from
the $\partial_{x}$ of equation \eqref{eq:mom_v}, yielding 
\begin{equation}
f\delta+\alpha\zeta=\zeta_{0}.\label{eq:delta}
\end{equation}
Using equation \eqref{eq:zeta} in equation \eqref{eq:delta} gives
us

\[
\delta=-\frac{1}{f^{2}+\alpha^{2}}\left(\alpha g\Delta\eta-\alpha\delta_{0}-f\zeta_{0}\right).
\]
Finally, substituting $\delta$ in equation \eqref{eq:mass}, that
reads $\bar{\eta}\delta+\alpha\eta=\eta_{0}$, results in 
\[
-\frac{\bar{\eta}}{f^{2}+\alpha^{2}}\left(\alpha g\Delta\eta-\alpha\delta_{0}-f\zeta_{0}\right)+\alpha\eta=\eta_{0},
\]
which may be simplified into the elliptic equation by multiplying
by $-\frac{f^{2}+\alpha^{2}}{\bar{\eta}\alpha g}$

\begin{equation}
\Delta\eta-\kappa^{2}\eta=r_{0}\label{eq:ellip}
\end{equation}
where 
\[
\kappa^{2}=\frac{f^{2}+\alpha^{2}}{\bar{\eta}g}
\]
and 
\[
r_{0}=-\frac{\kappa^{2}}{\alpha}\eta_{0}+\frac{1}{g}\delta_{0}+\frac{f}{\alpha g}\zeta_{0}.
\]


Multiplying the equation \eqref{eq:ellip}by $-g\bar{\eta}$ gives 

\[
((\alpha^{2}+f^{2})-g\bar{\eta}\Delta)\eta=\frac{f^{2}+\alpha^{2}}{\alpha}\eta_{0}-\bar{\eta}\delta_{0}-\frac{f\bar{\eta}}{\alpha}\zeta_{0}
\]


We can compute$\eta$ calculated from any elliptic solver. Special
attention has to be given to the LHS of the form $(\gamma-\Delta)\eta$
for a spectral solver and this is the reason for this brief excursion.
We use the annotation $\tilde{}$ to annotate an operator or quantity
to be given in spectral space. Then, we factor $\tilde{\eta}$ in
and write form of the LHS in spectral space with the identity matrix
$I$ as
\[
(\gamma I\tilde{\eta}-\tilde{\Delta}\tilde{\eta})=(\gamma I-\tilde{\Delta})\tilde{\eta}
\]
Here, the minus operator in spectral space has to be applied on all
elements of $\tilde{\Delta}$ and not only to the $0^{th}$ mode as
it is typically the case for adding a constant in spectral space.

We need to retrieve the velocities by solving the $2\times2$ system
formed by equations \eqref{eq:mom_u} and \eqref{eq:mom_v}, which
reads 
\[
A_{\alpha}U=U_{0}-g\nabla\eta
\]
with 
\[
A_{\alpha}=\left(\begin{matrix}\alpha & -f\\
f & \alpha
\end{matrix}\right).
\]
The solution is 
\[
U=A_{\alpha}^{-1}(U_{0}-g\nabla\eta)
\]
where 
\[
A_{\alpha}^{-1}=\frac{1}{f^{2}+\alpha^{2}}\left(\begin{matrix}\alpha & f\\
-f & \alpha
\end{matrix}\right)
\]
Finally, since we computed $(-L+\alpha)^{-1}$, we also have to invert
the sign of the computed solution $U$. Alternatively, we are allowed
to flip the signs of $\alpha_{n}$ and $\beta_{n}$.


\subsection{Interpretation of $\tau$}

We like to close this section with a brief discussion of $\tau$ by
having a look on the REXI reformulation

\[
(L-\frac{\alpha}{\tau})^{-1}U(\tau)\tau^{-1}=U(0)
\]
We see, that for an increasing $\tau$, hence an integration in time
over a larger time period, the poles given by $\alpha$ are getting
closer. This can possibly lead to a loss in accuracy for the data
sampled by the outer poles $\alpha_{-N}$ and $\alpha_{N}$. Therefore,
the number $N$ of poles is expected to scale linearly with the size
of the coarse time step, 
\[
|N|\propto\tau.
\]


Indeed, we saw in section \ref{sec:mat_exp} that for larger $\tau$,
$M$ needs to be larger.


\subsection{Complex solver}

We are interested in solving a system of the form, 
\[
(-L+\alpha)U=U_{0},
\]
where $\alpha$ is a complex number, therefore we must allow complex
solutions for $U$. The system can be transformed to have only real
arithmetic in the following way.

First we decompose $U$ into its real and imaginary parts, $U=U^{r}+iU^{i}$,
and allow $U_{0}$ to be decomposed in the same way. Although $\alpha$
can be a general complex number, it real part is always constant in
REXI, given by $\mu$. We will therefore write $\alpha=\mu+i\alpha^{i}$,
and we can absorb $\mu$ into $L$ writing $D=-L+\mu I$, where $I$
is the identity matrix. Now 
\[
(D+i\alpha^{i}I)(U^{r}+iU^{i})=U_{0}^{r}+iU_{0}^{i},
\]
\[
DU^{r}-\alpha^{i}U^{i}+i(\alpha^{i}U^{r}+DU^{i})=U_{0}^{r}+iU_{0}^{i},
\]
therefore 
\begin{eqnarray}
DU^{r}-\alpha^{i}U^{i} & = & U_{0}^{r}\\
\alpha^{i}U^{r}+DU^{i} & = & U_{0}^{i},
\end{eqnarray}
which in matrix notation gives 
\[
\left(\begin{matrix}D & -\alpha^{i}I\\
\alpha^{i}I & D
\end{matrix}\right)\left(\begin{matrix}U^{r}\\
U^{i}
\end{matrix}\right)=\left(\begin{matrix}U_{0}^{r}\\
U_{0}^{i}
\end{matrix}\right),
\]
or, 
\[
\left(\begin{matrix}-L+\mu I & -\alpha^{i}I\\
\alpha^{i}I & -L+\mu I
\end{matrix}\right)\left(\begin{matrix}U^{r}\\
U^{i}
\end{matrix}\right)=\left(\begin{matrix}U_{0}^{r}\\
U_{0}^{i}
\end{matrix}\right).
\]


A similar approach may be taken with the elliptic equation 
\begin{equation}
\Delta\eta+\theta\eta=r_{0},
\end{equation}
where $\eta$, $\theta$ and $r_{0}$ may be complex. The resulting
system is given by 
\begin{eqnarray}
\Delta\eta^{r}+\theta^{r}\eta^{r}-\theta^{i}\eta^{i}=r_{0}^{r},\\
\Delta\eta^{i}+\theta^{i}\eta^{r}+\theta^{r}\eta^{i}=r_{0}^{i},
\end{eqnarray}
which can also be written in matrix notation and solved with arbitrary
elliptic equation solvers.


\section{Filtering}

The method described in the previous section is well defined for skew
hermitian $L$. If $L$ is not skew hermitian, the real eigenvalues
might cause the REXI to have absolute values larger than 1, which
can lead to instabilities if used as time stepping method.

To ensure that the REXI is bounded by unit, a filtering process is
proposed in \cite{Terry:High-order time-parallel approximation of evolution operators}.
REXI is prone to exceed unit in the neighbourhood of $|t\lambda|\approx hM$,
therefore in the highest frequencies. The idea is to construct a rational
function $S(ix)$ that is approximately $1$ in a smaller interval
$|t\lambda|<hM_{0}$, with $M_{0}<M$, and decays very fast to zero
outside this interval. Then we multiply this filters function to the
original REXI, which will lead to a unit bounded REXI.

Further details of how $S(ix)$ is computed will be added later.


\section{Bringing everything together}

Using the spectral methods (e.g. in SWEET), we can directly solve
the Helmholtz problem for the height in Eq. (\ref{eq:helmhotz}) and
then solver for the velocity in Eqs. (\ref{BROKEN: Ref: eq:elliptic_velo...},\ref{eq:elliptic_velocity_v}).
Note that the Helmholtz problem is in complex space, as $\alpha_{n}$
are complex. This is straightforward with spectral methods. For finite
difference/element methods, the problem can be split into its real
and imaginary parts.

Then, the problem is reduced to computing the REXI as given in Eq.
\eqref{eq:rexi}. 


\section{Notes on HPC}
\begin{itemize}
\item The terms in REXI to solve are all independent. Hence, for latency
avoiding, the communication can be interleaved with computations. 
\item The iterative solvers are memory bound. Instead of computing $c:=a*b$
for the stencil operations, we could compute $\vec{c}:=a\vec{b}$
with $a$ one coefficient in the stencil. This allows vectorization
over $c$ and $b$ on accelerator cards with strided memory access. 
\item It is unknown which method is more efficient to solve the system of
equations:

\begin{itemize}
\item iterative solvers have low memory access, 
\item inverting the system and storing it as a sparse matrix allows fast
direct solving but can yield more memory access operations. 
\end{itemize}
\item Splitting the solver into real and complex number would store them
consecutively in memory. This has a potential to avoid non-strided
memory access and using the same SIMD operations (Just a rough idea,
TODO: check if this is really the case). 
\end{itemize}

\section{Acknowledgements}

Thanks to Terry for the feedback \& discussions! 
\begin{thebibliography}{1}
\bibitem{Schreiber:Formulations of the shallow-water equations}Formulations
of the shallow-water equations, M. Schreiber, P. Peixoto et al.

\bibitem{Terry:High-order time-parallel approximation of evolution operators}High-order
time-parallel approximation of evolution operators, T. Haut et al.

\bibitem{Terry:An asymptotic parallel-in-time method for highly oscillatory PDEs}An
asymptotic parallel-in-time method for highly oscillatory PDEs, T.
Haut et al.

\bibitem{Moler:Nineteen Dubious Ways to Compute the Exponential of a Matrix}Nineteen
Dubious Ways to Compute the Exponential of a Matrix, Twenty-Five Years
Later, Cleve Moler and Charles Van Loan, SIAM review

\bibitem{Damle:Near optimal rational approximations of large data sets}Near
optimal rational approximations of large data sets, Damle, A., Beylkin,
G., Haut, T. S. \& Monzon\end{thebibliography}

\end{document}
