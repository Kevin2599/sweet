%% LyX 2.1.3 created this file.  For more info, see http://www.lyx.org/.
%% Do not edit unless you really know what you are doing.
\documentclass[english]{article}
\usepackage[T1]{fontenc}
\usepackage[latin9]{inputenc}
%\usepackage{esint}
\usepackage{babel}
\usepackage{amsmath}
\usepackage{amsfonts}

\begin{document}

\title{Understanding the rational approximation of the exponential integrator
(REXI)}


\author{Martin Schreiber (M.Schreiber@exeter.ac.uk)\\
Pedro S. Peixoto (pedrosp@ime.usp.br)\\
et. al.}

\maketitle

%\begin{center}
%\textbf{!!!THIS IS A PRELIMINARY, NON PROOF-READ DOCUMENT!!!}
%\par
%\end{center}


This document serves as the basis for implementing the Rational approximation
of the EXponential Integrator (REXI). Here, we purely focus on the
linear part of the shallow-water equations (SWE) and show the different
steps to approximate solving this linear part with an exponential
integrator. This paper mainly summarises previous work on REXI.


\section{Problem formulation}

We use linearised shallow water equations (SWE) with respect to a rest state with mean water depth of $H$ and defined for perturbations of height $h$ (see \cite{Formulations of the shallow-water equations}). The linear operator ($L$) may be written as

%the advective formulation of the SWE with a full linearisation
%$with perturbation 
%with $U:=(h,u,v)^{T}$, yielding

\[
L(U):=\left(\begin{array}{ccc}
0 & H\delta_{x} & H\delta_{y}\\
g\delta_{x} & 0 & -f\\
g\delta_{y} & f & 0
\end{array}\right)U
\]
where $U:=(h,u,v)^{T}$. Here, we neglect all non-linear terms and consider $f$ constant (f-plane approximation). 
%, assume only small perturbations
%(hence negligible ones) around the average surface height and negligible
%non-linear terms.
% PP: We simply using the linear shallow water equations. 

The time evolution of the PDE, with the subscript $t$ denoting
the derivative in time, is given by

\[
U_{t}=L(U).
\]


%We continue writing the linear operator in matrix-style $L$ and applying
%$L$ on $U$ as $L.U$.
%
%\[
%U_{t}:=L.U
%\]
% PP : Not of any use

It is further worth noting, that this system describes an oscillatory
system (2D wave equation), hence the operator $L$ is hyperbolic and has imaginary eigenvalues.


%\section{Exponential integrator}

Linear initial value differential problems are well known to be solvable with exponential
integrators for arbitrary time step sizes via
\[
U(t)=e^{Lt}U(0).
\]
see e.g. \cite{Nineteen Dubious Ways to Compute the Exponential of a Matrix}.
However, this is typically quite expensive to compute and analytic
solutions only exist for some simplified system of equations, see
e.g. \cite{Formulations of the shallow-water equations} for f-plane
shallow-water equations. These exponential integrators can be approximated
with rational functions and this paper is on giving insight into this
approximation.


\section{1D rational approximation}

Terry et. al. \cite{High-order time-parallel approximation of evolution operators}
developed a rational approximation of the exponential
integrator.
First, we like to get more insight into it with a one-dimensional
formulation before applying REXI to a rational approximation of a
linear operator. Our main target is to find an approximation of an
operator with a \emph{complex exponential shape}, in our case $e^{ix}$,
which (in one-dimension) is given as a function $f(x)$. We will end
up in an approximation given by the following rational approximation:

\[
e^{ix}\approx\sum_{n=-N}^{N}\frac{\beta_{n}}{ix-\alpha_{n}}
\]
with complex coefficients $\alpha_{n}$ and $\beta_{n}$. We point out that the coefficients $\alpha_{n}$ will always have non zero real part, so no singularity occurs with the rational function.


\subsection{Step A: Approximation of solution space}

First, we assume that we can use Gaussian curves as basis functions
for our approximation 
%and they are somehow naturally related to this
%problem (Gaussians are given by exponential functions). 
% PP: The Gaussian has totally different structure compared with exp(ix). It is used because it has well known fourier transformation and localised support.
So first we find an approximation of one of our underlying Gaussian basis function

\[
\psi_{h}(x):=(4\pi)^{-\frac{1}{2}}e^{-x^{2}/(4h^{2})}
\]


In this formulation, $h$ can be interpreted as the horizontal ``stretching''
of the basis function. Note the similarities to the Gaussian distribution,
but by dropping certain parts of the vertical scaling as it is required
for probability distributions. We can now approximate our function
$f(x)$ with a superposition of basis functions $\psi_{h}(x)$ by

\[
f(x)\approx\sum_{m=-M}^{M}b_{m}\psi_{h}(x+mh)
\]
with $M$ controlling the interval of approximation (\textasciitilde{}size
of  ``domain of interest'') and $h$ will be related to the accuracy of integration (\textasciitilde{}resolution in  ``domain of interest''). 
%and the max. frequency of the oscillations generated
%by the linear operator. 
% PP: Needs better explanation

We choose $h$ small enough so that the support of the Fourier transform of $f$ is mainly localised within $[-1/(2h),1/(2h)]$, i.e. almost zero outside this interval. $M$ is chosen such that the approximation will be adequate in the interval $|x|<Mh$. 

%$M$ and $h$ are connected in the sense that to obtain a good approximation for a larger interval, $M$ needs to be larger and $h$ smaller.

 To compute the coefficients $b_{m}$, we rewrite
the previous equation in Fourier space with
\[
\frac{\hat{f}(\xi)}{\hat{\psi_{h}}(\xi)}=\sum_{m=\infty}^{\infty}b_{m}e^{2\pi imh\xi},
\]
where the $\hat{\cdot}$ symbols indicate the Fourier transforms of the respective functions. The $b_m$ are now the Fourier coefficients of the series for the function $\frac{\hat{f}(\xi)}{\hat{\psi_{h}}(\xi)}$ and can be calculated as \footnote{see \cite{High-order time-parallel approximation of evolution operators},
page 11}, 
\[
b_{m}=h\intop_{-\frac{1}{2h}}^{\frac{1}{2h}}e^{-2\pi imh\xi}\frac{\hat{f}(\xi)}{\hat{\psi_{h}}(\xi)}d\xi,
\]
for $m\in \mathbb{Z}$ and $1/h$ defines the periodicity of the trigonometric basis function. 

Since we are interested in approximating $f(x)=e^{ix}$, we
can simplify the equation by using the response in frequency space
$\hat{f}(\xi)=\delta(\xi-\frac{1}{2\pi})$, where here $\delta$ is the Dirac distribution, and

\[
b_{m}=h\,e^{-imh}\hat{\psi_{h}}(\frac{1}{2\pi})^{-1}.
\]

%PP : you cannot write $\hat{f}(\frac{1}{2\pi})$, as this is infinity at this point. But can calculate integral around this point and the final result was valid.
%\[
%b_{m}=h\,e^{-2\pi imh\frac{1}{2\pi}}\frac{\hat{f}(\frac{1}{2\pi})}{\hat{\psi_{h}}(\frac{1}{2\pi})}=h\,e^{-imh}\hat{\psi_{h}}%(\frac{1}{2\pi})^{-1}.
%\]

The Fourier transform of the Gaussian function is well known and given by

\[
\hat{\psi_{h}}(\xi)=\intop_{-\infty}^{\infty}\frac{1}{\sqrt{4\pi}}e^{-\left(\frac{x}{2h}\right)^{2}}e^{-2\pi ix\xi}dx =he^{-(2h\pi\xi)^{2}} 
\]
%\[
%=\frac{1}{\sqrt{4\pi}}\intop_{-\infty}^{\infty}e^{-\left(\left(\frac{x}{2h}\right)^{2}+2\pi ix\xi+(2h\pi i\xi)^{2}-(2h\pi i\xi)^{2}\right)}dx
%\]
%\[
%=\frac{1}{\sqrt{4\pi}}e^{-(2h\pi\xi)^{2}}\intop_{-\infty}^{\infty}e^{-\left(\frac{x}{2h}+2h\pi i\xi\right)^{2}}dx
%\]
%
%\[
%=\frac{1}{\sqrt{4\pi}}e^{-(2h\pi\xi)^{2}}\intop_{-\infty}^{\infty}e^{-\left(\frac{x}{2h}\right)^{2}}dx
%\]
where we used that $\intop_{-\infty}^{\infty}e^{-\left(\frac{x}{2h}\right)^{2}}dx=h\,\sqrt{4\pi}$ and completed squares in the exponential term. For the case $\xi=\frac{1}{2\pi}$, we get

\[
\hat{\psi}_{h}\left(\frac{1}{2\pi}\right)=h\,e^{-h^{2}}.
\]
Finally, one can obtain the equation

\[
b_{m}=h\,e^{-imh}\frac{1}{h\,e^{-h^{2}}}=e^{-imh}e^{h^{2}}
\]
to compute the coefficients $b_{m}$ for $f(x)=e^{ix}$.


\subsection{Step B: Approximation of basis function}

The second step is the approximation of the basis function $\psi_{h}(x)$
itself with a rational approximation, see \cite{Near optimal rational approximations of large data sets}.
Our basis function is given by

\[
\psi_{h}(x):=(4\pi)^{-\frac{1}{2}}e^{-x^{2}/(4h^{2})}
\]
and a close-to-optimal approximation of $\psi_{1}(x)$ with a sum
of rational functions is given by

\[
\psi_{1}(x)\approx Re\left(\sum_{l=-L}^{L}\frac{a_{l}}{ix+(\mu+i\,l)}\right)
\]
with the $\mu$ and $a_{l}$ given in \cite{Near optimal rational approximations of large data sets},
Table 1. We can generalise this approximation to arbitrary chosen
$h$ via 
\[
\psi_{h}(x)\approx Re\left(\sum_{l=-L}^{L}\frac{a_{l}}{i\frac{x}{h}+(\mu+i\,l)}\right).
\]

The theory of how these coefficients are calculated are in \cite{Near optimal rational approximations of large data sets} and will not be described here. Therefore, we assume that the coefficients $a_l$ are given.

\subsection{Step C: Approximation of the approximation}

We then combine the approximation (B) into the approximation (A), yielding the approximation $\tilde{f}$ for $f$ given by
\[
\tilde{f}(x)=\sum_{m=-M}^{M}b_{m}\psi_{h}(x+mh)=\sum_{m=-M}^{M}b_{m}Re\left(\sum_{l=-L}^{L}\frac{a_{l}}{i\frac{x+mh}{h}+(\mu+i\,l)}\right)
\]


\[
=\sum_{m=-M}^{M}b_{m}\sum_{l=-L}^{L}Re\left(\frac{ha_{l}}{ix+h(\mu+i(m+l))}\right).
\]

We can simplify the summations assuming $n=m+l$ and inverting their order,
\begin{eqnarray*}
\tilde{f}(x)& =& \sum_{l=-L}^{L} \, \sum_{m=-M}^{M}b_{m}Re\left(\frac{ha_{l}}{ix+h(\mu+i(m+l))}\right) \\
& =& \sum_{l=-L}^{L}\,\sum_{n=-M+l}^{M+l}b_{n-l}Re\left(\frac{ha_{l}}{ix+h(\mu+in)}\right) \\
& =& \sum_{n=-N}^{N}\,\sum_{k=L_1}^{L_2}b_{n-k}Re\left(\frac{ha_{k}}{ix+h(\mu+in)}\right),
\end{eqnarray*}
where $N=L+M$, $L_1=max(-L, n-M)$ and $L_2=min(L,n+M)$.

%We further like to simplify this equation and we observe, that for
%$n:=m+l$, the denominator is equal. We can hence express parts of
%the denominator in terms of $n$ by

We will define the poles as
\begin{equation}
\alpha_{n}:=h(\mu+i n).
\label{eq:alpha}
\end{equation}

For the real part of $f(x)$ we have that
\begin{eqnarray*}
Re(\tilde{f}(x))& =& \sum_{n=-N}^{N}\,\sum_{k=L_1}^{L_2}Re(b_{n-k})Re\left(\frac{ha_{k}}{ix+\alpha_{n}}\right) \\
& =& Re\left(\sum_{n=-N}^{N}\, \sum_{k=L_1}^{L_2}Re(b_{n-k})\frac{ha_{k}}{ix+\alpha_{n}}\right) \\
& =& Re\left(\sum_{n=-N}^{N}\frac{1}{ix+\alpha_{n}}\, h \sum_{k=L_1}^{L_2}Re(b_{n-k})a_{k}\right).
\end{eqnarray*}
Defining the residue as 
\begin{equation}
\beta^{Re}_{n}:=h \sum_{k=L_1}^{L_2}Re(b_{n-k})a_{k},
\label{eq:beta_re}
\end{equation}
we have that 
\[
Re(\tilde{f}(x))=Re\left(\sum_{n=-N}^{N}\frac{\beta^{Re}_n}{ix+\alpha_{n}}\right).
\]

Therefore, if $f$ is in fact real, we should be able to directly apply
\[
\tilde{f}(x)=\sum_{n=-N}^{N}\frac{\beta_n}{ix+\alpha_{n}},
\]
where we have dropped the index in the $\beta$ coefficients, but explicitly state that they are given by
\begin{equation}
\beta_{n}:=h \sum_{k=L_1}^{L_2}Re(b_{n-k})a_{k}.
\label{eq:beta}
\end{equation}

If $f$ is complex, its imaginary part can be analogously found by
\[
Im(\tilde{f}(x)) = Re\left(\sum_{n=-N}^{N}\frac{\beta_n^{Im}}{ix+\alpha_{n}}\right),
\]
where 
\[
\beta^{Im}_{n}:=h \sum_{k=L_1}^{L_2}Im(b_{n-k})a_{k}.
\]

%Now, we merge the $b_{m}$ and $a_{l}$ coefficients and first have
%a look at the $b_{m}$ which is complex values. We observe the following
%property: Assuming that we want to compute the real value of $f(x)$,
%only the real value of $b_{m}$ has to be merged with the sum, since
%the imaginary component would be dropped afterwards. This allows us
%to move the $Re(b_{m})$ values inside the $\sum_{L}$:
%
%\[
%Re(f(x)):=Re\left(\sum_{m=-M}^{M}\,\,\sum_{l=-L}^{L}\frac{Re(b_{m})\,a_{l}}{ix+s(\mu+i(m+l))}\right).
%\]
%Now we can collect all nominators with equivalent denominator (if
%$n=m+l$ and by using $\delta$ as the Kronecker delta), yielding
%
%\[
%\beta_{n}^{Re}:=\sum_{m=-M\,}^{M}\sum_{l=-L}^{L}Re(b_{m})a_{l}\delta(n,\,m+l)
%\]
%for real values $f(x)$ and
%
%\[
%\beta_{n}^{Im}:=\sum_{m=-M\,}^{M}\sum_{l=-L}^{L}Im(b_{m})a_{l}\delta(n,\,m+l)
%\]
%for complex values of $f(x)$. 

This finally leads us to the REXI approximation

\[
e^{ix}\approx\sum_{n=-N}^{N}Re\left(\frac{\beta_{n}^{Re}}{ix+\alpha_{n}}\right)+i\,Re\left(\frac{\beta_{n}^{Im}}{ix+\alpha_{n}}\right)
\]
for the complex-valued function $e^{ix}$.


\section{REXI on linear operators}

\subsection{Matrix exponential}
\label{sec:mat_exp}
We would like to apply REXI to a formulation such as 
\[
U(t)=e^{tL}U(0).
\]
To see the relationship between the approximation of $e^{ix}$ with $e^{tL}$ we assume that $L$ is skew hermitian and therefore has only purely imaginary eigenvalues, and maybe decomposed as $\Sigma\Lambda\Sigma^{H}$, yielding
\[
e^{tL}=\sum_{k=0}^{\infty}\frac{tL}{k!}=\Sigma\left(\sum_{k=0}^{\infty}\frac{t\Lambda}{k!}\right)\Sigma^H=\Sigma e^{t\Lambda}\Sigma^H,
\]
where we used the orthonormality of $\Sigma$ to remove it from the summation, and

\[
e^{t\Lambda}=\left(\begin{array}{ccc}
...\\
 & e^{i\lambda_{n}t}\\
 &  & ...
\end{array}\right),
\]
where we hare explicitly detached the imaginary unit from the eigenvalues, therefore $\lambda_n$ are assumed real. Since $e^{t\Lambda}$ is diagonal, it can be eigenvalue-wise approximated in the same way as in $e^{ix}$.

Some important points about the choice of $M$ and $h$ have to be made at this point. We know that $e^{ix}$ is accurately approximated with REXI for the interval $|x| < hM$, where $h$ is chosen small enough to obtain a good approximation in step (A), and $M$ will define the interval size and number of approximation points. In the matrix case, $M$ has to be chosen so that $ hM > t\bar{\lambda}$, where $\bar{\lambda}=\max_{n}|\lambda_n|$, in order to capture all wavelengths of $L$. In other other words, $hM$ need to be set to capture the fastest wave. Note that if this is used as a time stepping method, with time step $t=\tau$, then, the larger the timestep, the larger $M$ will be. Exact evaluations of the choices for $h$ and $M$ may be done based on equation (3.6) of \cite{High-order time-parallel approximation of evolution operators}.


%The choice of $M$ is related to the eigen-structure of $L$. We will briefly discuss how this works. the We will eventually substitute $e{ix}$ with $e^{\tauL}$ in Now we need to chose $M$ in a way to ensure that $\tau \Lambda \in [-Mh, Mh]$, where $\Lambda$ is such that all eigenvalues $\lambda$ of $L$ satisfy $|\lambda| < \Lambda $. That is, the domain of interest must capture waves of frequency up until $\Lambda$.

%
%with complex-valued exponentials on the eigenvalues. Hence, the accuracy
%of the exponential integrator on $f(L)$ only depends on the spectrum
%of the $L$ and allows to be applied in the same way as $e^{ix}$,
%but by replacing $x$ with the matrix $L$. For error bounds, we like
%to refer to \cite{High-order time-parallel approximation of evolution operators}.


%However, the interpolated values can exceed this unity due to interpolation
%properties (think of a Lagrangian interpolation of high order, leading
%to large oscillations with the possibility of exceeding the local
%min/max of the interpolated function in the area of support). This
%can lead to long-term effects such as amplifying unphysical solutions.
%Therefore a filtering may be required to assure that the function
%in the interpolation range is always bounded by unity.



We want to evaluate $e^{\tau L}U(0)$ with REXI, where $\tau$ will be a time step size and $U(0)$ the initial condition for this time step.  We will assume $\tau$ a-priori fixed, which implies that the coefficients in REXI will not change and may be pre-computed.

%will assume 
%In the following, we use $L:=\tau L'$ and assume an a-priori fixed
%time step size. Fixing a time step size will make a REXI approximation more efficient. Then,
%the REXI approximation is given by

Although $L$ has imaginary eigenvalues, we wish to evaluate the $e^{\tau L}U(0)$, which is real valued, therefore, we will use the real approximation of $e^{ix}$
\begin{equation}
exp(\tau L) \approx Re \left( \sum_{n=-N}^{N}\beta_{n}(\tau L-\alpha_{n})^{-1} \right),
\label{eq:rexi}
\end{equation}
where $\beta_n$ is given by equation (\ref{eq:beta_re}) and $\alpha_n$ by equation (\ref{eq:alpha}). These coefficient may be pre-computed if $L$ and $\tau$ are fixed.


%The coefficients $\alpha_{k}$ (corresponding to $s(\mu+i\,n)$ in
%step C for the one-dimensional formulation) can be precomputed or
%computed during program start. $\mu$ is based on a one-dimensional
%approximation, see the paper, and the $\alpha_{k}$ can be interpreted
%as shifts of the rational approximations. The coefficients $\beta_{k}$
%(corresponding to $c_{n}$ in step C) are describing the scaling of
%the basis function and are also constant and independent of the solution
%itself. 

Note, that for debugging purpose, their \emph{imaginary values
have to cancel out}. (PP: are you sure?)

Note an important property (see Sec. 3.3 in \cite{High-order time-parallel approximation of evolution operators}).
There's an anti-symmetry in the $\alpha_{i}$ coefficients, which
avoids computing half of the inverses,
\[
\overline{(L-\alpha)^{-1}U(0)}=(L-\overline{\alpha})^{-1}U(0).
\]





\subsection{Handling $\tau$ in REXI\label{sub:Handling-tau-in-REXI}}

%We recall the formulation of the solution as an exponential integrator
%
%\[
%U(t)=e^{tL}U(0)
%\]
%which formally allows us to join the integration in time given by
%$t$ with the $L$ operator in case of such a formulation. 

%There are
%basically two different ways to handle this scaling:

%The first one is rescaling all parameters by $\tau$:

%\[
%g':=\tau g,\,\,\,\,\,f':=\tau f,\,\,\,\,\,h_{0}':=\tau h_{0}.
%\]


%The second way is to 
We reformulate the REXI approximation scheme given
by

\[
(\tau L-\alpha)^{-1} U(\tau)=U(0)
\]
and by factoring $\tau$ out, yielding

\[
(L-\frac{\alpha}{\tau})^{-1} U(\tau)\tau^{-1}=U(0)
\]
So instead of solving for $U(\tau)$, we are solving for $U^{\tau}(\tau):=U(\tau)\tau^{-1}$
as well as $\alpha^{\tau}:=\frac{\alpha}{\tau}$.

To summarize, we have to solve the system of equations given by

\begin{equation}
(L-\alpha^{\tau})^{-1} U^{\tau}(\tau)=U(0)\label{eq:unit_rexi_timestep}
\end{equation}
with $U(0)$ the initial conditions. For sake of simplicity, we stick
to the formulation without the $\tau$ notation.
%
%One final scaling has to be done: the exponential is computing $e^{\tau L}$,
%hence the real-valued $\tau$ has to be included in the operator $L$.
%There are basically two different ways: The first one is rescaling
%all parameters by $\tau$:
%
%\[
%g^{\tau}:=\tau g,\,\,\,\,\,f^{\tau}:=\tau f,\,\,\,\,\,h_{0}^{\tau}:=\tau h_{0}
%\]
%The second way is to factor the $\tau$ parameter out:
%
%\[
%\left(\tau L-\alpha\right)^{-1}.U(\tau)=U(0)
%\]
%
%
%\[
%\left(L-\frac{\alpha}{\tau}\right)^{-1}.U(\tau)\tau^{-1}=U(0)
%\]
%So instead of solving for $U(\tau)$, we are solving for $U^{\tau}(\tau):=U(\tau)\tau$
%as well as $\alpha^{\tau}:=\frac{\alpha}{\tau}$and have to divide
%the computed solution by $\tau$ in the end.


\subsection{Computing inverse of $(L-\alpha)^{-1}$}

For computing the inverse, arbitrary solvers can be used. However
we like to note, that $\alpha$ is a complex number. Hence, requiring
solvers with support for solving in complex space.

% As an example,
%we consider a specialization on the shallow-water equations given
%above with

%\[
%L(U(t)):=\left(\begin{array}{ccc}
% & H\delta_{x} & H\delta_{y}\\
%g\delta_{x} &  & -f\\
%g\delta_{y} & f
%\end{array}\right)U(t)
%\]
%\[
%U_{t}(t):=L(U(t)).
%\]
%and we set $g:=1$ and the average height $H:=1$. 


We wish to solve the differential problem for each time step
\[ (L-\alpha)U=U_0 \] so that $U=(L-\alpha)^{-1}U_0$. We will do this converting the problem into and elliptic equation.


First, lets expand the equations with the definition of $L$,
\begin{eqnarray}
 -f v - g\eta_x -\alpha u &=& u_0 ,
 \label{eq:mom_u}\\
f u - g\eta_y -\alpha v &=& v_0, 
 \label{eq:mom_v} \\
 H ( u_x+v_y) -\alpha \eta  &=& \eta_0.
 \label{eq:mass}
\end{eqnarray}

Let $f$ be constant (f-plane approximation), $\delta:=u_x+v_y$ be the wind divergence, $\zeta:=v_x-u_y$ be the wind (relative) vorticity and $\Delta \eta:= \eta_{xx}+\eta_{yy}$ the Laplacian of the fluid depth. We will re-write the problem in a divergence-vorticity formulation by taking 2 steps. First, sum the $\partial_x$ of equation (\ref{eq:mom_u}) and the $\partial_y$ of equation (\ref{eq:mom_v}), yielding
\begin{equation}
-f\zeta -g \Delta \eta - \alpha \delta = \delta _0.
\label{eq:zeta}
\end{equation}
Then subtract the $\partial_y$ of equation (\ref{eq:mom_u}) from the $\partial_x$ of equation (\ref{eq:mom_v}), yielding
\begin{equation}
f\delta - \alpha \zeta = \zeta_0.
\label{eq:delta}
\end{equation}
Using equation (\ref{eq:zeta}) in equation (\ref{eq:delta}) gives us

\[
\delta =- \frac{1}{f^ 2+\alpha^ 2} \left( \alpha g \Delta \eta + \alpha \delta _0  - f\zeta_0 \right).
\]

Finally, substituting $\delta$ in equation (\ref{eq:mass}), that reads  $ H \delta -\alpha \eta  = \eta_0$, results in
\[
- \frac{H}{f^ 2+\alpha^ 2} \left( \alpha g \Delta \eta + \alpha \delta _0  - f\zeta_0 \right) -\alpha \eta  = \eta_0 ,
\]
which may be simplified into the elliptic Helmholtz equation by multiplying by $ - \frac{f^ 2+\alpha^ 2}{H\alpha g}$

\begin{equation}
\Delta \eta  + \kappa^2 \eta  = r_0
\label{eq:hemholtz}
\end{equation}
where \[ \kappa^2=\frac{f^2+\alpha^2}{Hg} \] and 
\[
r_0=-\frac{\kappa^2}{\alpha}\eta_0 -\frac{1}{g}\delta_0 + \frac{f}{\alpha g} \zeta_0. 
\]


Once $\eta$ is calculated from any Helmholtz solver, we need to retrieve the velocities by solving the $2\times 2$ system formed by equations (\ref{eq:mom_u}) and (\ref{eq:mom_v}), which reads
\[
A_{\alpha}U=U_0+g\nabla \eta
\]
with
\[A_{\alpha}=
\left( \begin{matrix}
-\alpha & -f \\
f & -\alpha 
\end{matrix}
\right).
\]
The solution is
\begin{equation}
u=-\frac{1}{f^2+\alpha^2}\left[g(\alpha\eta_x-f\eta_y)+\alpha u_0-fv_0 \right]
\label{eq:elliptic_velocity_u}
\end{equation}
and
\begin{equation}
v=-\frac{1}{f^2+\alpha^2}\left[g(f\eta_x+\alpha\eta_y)+f u_0+\alpha v_0 \right],
\label{eq:elliptic_velocity_v}
\end{equation}
which may be written as
\[
U=A^ {-1}_{\alpha}(g\nabla \eta+U_0)	
\]
where
\[
A^ {-1}_{\alpha}=-\frac{1}{f^2+\alpha^2}
\left( \begin{matrix}
\alpha & -f \\
f & \alpha 
\end{matrix}.
\right)
\]

%We like to mention again, that we can use arbitrary solvers and in
%this work, we focus on a reformulation into an eliptic problem. Following
%the idea in \cite{An invariant theory of the linearized shallow water equations with rotation and its application to a sphere and a plane},
%instead of solving this relatively large system of equations we can
%split the problem into an elliptic one for the height which then allows
%to use an explicit formulation for the velocities. We use the abbreviation$\vec{v}:=(u,v)$
%in the following paragraph and the formulation with a unit time step
%(see Eq. \ref{eq:unit_rexi_timestep}). Using the formulation in \cite{High-order time-parallel approximation of evolution operators},
%the height can be computed with the elliptic equation given by
%
%\begin{equation}
%(\nabla^{2}-(\alpha^{2}+f^{2}))h(1)=\frac{\alpha^{2}+f^{2}}{\alpha}(h(0)+H\nabla\cdot(A\,\vec{v}(0))\label{eq:elliptic_height}
%\end{equation}
%with
%
%\[
%A:=\frac{1}{\alpha^{2}+f^{2}}\left(\begin{array}{cc}
%\alpha & -f\\
%f & \alpha
%\end{array}\right).
%\]
%

%
% \subsection{f-plane}
%
%Assuming an f-plane approximation (f is constant), we can rearrange
%this equation by using the abbreviations $\kappa:=\alpha^{2}+f^{2}$
%in the following way:
%
%\[
%(\nabla^{2}-\kappa)\,h(1)=\frac{\kappa}{\alpha}\left(h(0)+\nabla\cdot(A\,\vec{v}(0)\right)
%\]
%
%
%\[
%(\nabla^{2}-\kappa)\,h(1)=\frac{\kappa}{\alpha}h(0)+\frac{1}{\alpha}\nabla\cdot\left(\begin{array}{cc}
%\alpha & -f\\
%f & \alpha
%\end{array}\right)\vec{v}(0)
%\]
%
%
%\begin{equation}
%(\nabla^{2}-\kappa)h(1)=\frac{\kappa}{\alpha}h(0)-\frac{f}{\alpha}\nabla\times v(0)+\nabla\cdot\vec{v}(0)
%\end{equation}
%Here, the $\alpha$ and $\kappa$ denote the terms with imaginary
%numbers and this formulation should also simplify programming.
%
%We continue with an interpretation of this formulation: on the right
%hand side we see an update-like scheme $h(0)$ in the first scheme,
%then a vorticity-like formulation $\times$, and an advective part
%$\nabla$. To simplify the notation for solving the system, we rewrite
%it as
%
%\begin{equation}
%(\nabla^{2}-\kappa)\,h(1)=D
%\end{equation}
%with real-and-imaginary-valued $D$ and $\nabla^{2}-\kappa$ as well
%as a real-and-imaginary-valued $h(1)$ for which we want to solve.
%Then, the solution is given e.g. in spectral space directly via
%\[
%h(1):=D(\nabla^{2}-\kappa)^{-1}
%\]
%Once computed the height, the velocities can be directly computed
%via
%
%\begin{equation}
%\vec{v}(1)=-A.\vec{v}(0)+A.\nabla h(1)=-A.(\vec{v}(0)-\nabla h(1))\label{eq:elliptic_velocity}
%\end{equation}
%giving us our final solution
%
%\[
%U(\tau):=\tau(h(1),\,u(1),\,v(1))^{T}
%\]
%with the scaling with $\tau$ as discussed in Sec. \ref{sub:Handling-tau-in-REXI}
%and we like to mention, that also $\alpha$ has to be scaled appropriately
%before using it for REXI.


\subsection{Interpretation of $\tau$}

We like to close this section with a brief discussion of $\tau$ by
having a look on the REXI reformulation

\[
(L-\frac{\alpha}{\tau})^{-1} U(\tau)\tau^{-1}=U(0)
\]
We see, that for an increasing $\tau$, hence an integration in time
over a larger time period, the poles given by $\alpha$ are getting
closer. This can possibly lead to a loss in accuracy for the data
sampled by the outer poles $\alpha_{-N}$ and $\alpha_{N}$. Therefore,
the number $N$ of poles is expected to scale linearly with the size
of the coarse time step,
\[
|N| \propto \tau.
\]

Indeed, we saw in section \ref{sec:mat_exp} that for larger $\tau$, $M$ needs to be larger!

%{[}TODO (Terry): There's probably a tighter relationship somewhere
%hidden in the paper{]}


\section{Filtering}

The method described in the previous section is well defined for skew hermitian $L$. If $L$ is not skew hermitian, the real eigenvalues might cause the REXI to have absolute values larger than 1, which can lead to instabilities if used as time stepping method.

To ensure that the REXI is bounded by unit, a filtering process is proposed in \cite{High-order time-parallel approximation of evolution operators}. REXI is prone to exceed unit in the neighbourhood of $| t \lambda | \approx hM$, therefore in the highest frequencies. The idea is to construct a rational function $S(ix)$ that is approximately $1$ in a smaller interval   $  | t \lambda | < hM_0 $, with $M_0<M$, and decays very fast to zero outside this interval. Then we multiply this filters function to the original REXI, which will lead to a unit bounded REXI.

Further details of how $S(ix)$ is computed will be added later.

\section{Bringing everything together}

Using the spectral methods (e.g. in SWEET), we can directly solve
the Helmholtz problem for the height in Eq. (\ref{eq:hemholtz}) and then solver for the
velocity in Eqs. (\ref{eq:elliptic_velocity_u},\ref{eq:elliptic_velocity_v}). Note that the Helmholtz problem is in complex space, as $\alpha$ is complex. this is straightforward with spectral methods. for finite difference/element methods, the problems needs to be split into its real and imaginary parts.

Then, the problem is reduced to computing the REXI as given in Eq. (\ref{eq:rexi}). We
like to note again, that the $\alpha_{n}$ and $\beta_{n}$ are independent
of the system $L$ to solve, and the number of coefficients only depends
on the accuracy and the resolution.



\section{Notes on HPC}
\begin{itemize}
\item The terms in REXI to solve are all independent. Hence, for latency
avoiding, the communication can be interleaved with computations.
\item The iterative solvers are memory bound. Instead of computing $c:=a*b$
for the stencil operations, we could compute $\vec{c}:=a\vec{b}$
with $a$ one coefficient in the stencil. This allows vectorization
over $c$ and $b$ on accelerator cards with strided memory access.
\item It is unknown which method is more efficient to solve the system of
equations:

\begin{itemize}
\item iterative solvers have low memory access,
\item inverting the system and storing it as a sparse matrix allows fast
direct solving but can yield more memory access operations.
\end{itemize}
\item Splitting the solver into real and complex number would store them
consecutively in memory. This has a potential to avoid non-strided
memory access and using the same SIMD operations (Just a rough idea,
TODO: check if this is really the case).
\end{itemize}

\section{Acknowledgements}

Thanks to  Terry for the feedback \& discussions!
\begin{thebibliography}{1}
\bibitem{Formulations of the shallow-water equations}Formulations
of the shallow-water equations, M. Schreiber

\bibitem{High-order time-parallel approximation of evolution operators}High-order
time-parallel approximation of evolution operators, T. Haut et. al.

\bibitem{An asymptotic parallel-in-time method for highly oscillatory PDEs}An
asymptotic parallel-in-time method for highly oscillatory PDEs, T.
Haut et. al.

\bibitem{An invariant theory of the linearized shallow water equations with rotation and its application to a sphere and a plane}An
invariant theory of the linearized shallow water equations with rotation
and its application to a sphere and a plane, N. Paldor et. al.

\bibitem{Nineteen Dubious Ways to Compute the Exponential of a Matrix}Nineteen
Dubious Ways to Compute the Exponential of a Matrix, Twenty-Five Years
Later, Cleve Moler and Charles Van Loan, SIAM review

\bibitem{Near optimal rational approximations of large data sets}Near
optimal rational approximations of large data sets, Damle, A., Beylkin,
G., Haut, T. S. \& Monzon\end{thebibliography}

\end{document}
