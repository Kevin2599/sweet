%% LyX 2.1.3 created this file.  For more info, see http://www.lyx.org/.
%% Do not edit unless you really know what you are doing.
\documentclass[english]{article}
\usepackage[T1]{fontenc}
\usepackage[latin9]{inputenc}
\PassOptionsToPackage{normalem}{ulem}
\usepackage{ulem}

\makeatletter

%%%%%%%%%%%%%%%%%%%%%%%%%%%%%% LyX specific LaTeX commands.
%% Because html converters don't know tabularnewline
\providecommand{\tabularnewline}{\\}

\makeatother

\usepackage{babel}
\begin{document}

\title{Formulations of the shallow-water equations}


\author{Martin Schreiber <M.Schreiber@exeter.ac.uk>}


%\date{\emph{2015-07-13}}
%\date{}

\maketitle
There are a variety of different formulations for the shallow-water
equations available. This document serves as an overview of these
different formulations and how they are approximated.


\section{Overview}

We will adopt the following terminology:
\begin{tabular}{|c|l|}
\hline 
 Symbol & Definition \tabularnewline
\hline 
\hline 
$h$ & Fluid layer depth/height\tabularnewline
\hline 
$u$ & Velocity in x direction\tabularnewline
\hline 
$v$ & Velocity in y direction\tabularnewline
\hline 
$hu$ & Momentum in x direction\tabularnewline
\hline 
$hv$ & Momentum in y direction\tabularnewline
\hline 
$\eta$ & Potential vorticity\tabularnewline
\hline 
$g$ & Gravity\tabularnewline
\hline 
$H$ & Bernoulli potential\tabularnewline
\hline 
$f$ & Coriolis frequency \tabularnewline
\hline 
$p_{0}$ & Steady state of quantity\tabularnewline
\hline 
$p'$ & Perturbation of $p_{0}$\tabularnewline
\hline 
\end{tabular}

We will also assume no bottom topography (orography/bathymetry), so the fluid layer height is the total height. 

\section{Momentum formulation}

The conservative formulation uses the momentum $(hu,hv)$ as the conserved
quantities. In this form, the equations are given as

\[
h_{t}+(hu)_{x}+(hv)_{y}=0
\]


\[
(hu)_{t}+(hu^{2}+\frac{1}{2}gh^{2})_{x}+(huv)_{y}-fv=0
\]


\[
(hv)_{t}+(hv^{2}+\frac{1}{2}gh^{2})_{y}+(huv)_{x}+fu=0
\]


Since this formulation is conserving the momentum, it is also refered
as the conservative form.

Please note, that this formulation contains several non-linear terms!


\section{Non-conservative formulation}

By using the produce rule, we can reformulate the equations above
to make the velocities conserved quantities. Here, we first apply
the produce rule to $(hu)_{t}$

\[
hu_{t}+h_{t}u+h_{x}u^{2}+2huu_{x}+ghh_{x}+(hu)_{y}v+huv_{y}-fv=0
\]


\[
hu_{t}+(-(hu)_{x}-(hv)_{y})u+h_{x}u^{2}+2huu_{x}+ghh_{x}+(hv)_{y}u+hvu_{y}-fv=0
\]


\[
hu_{t}+(-(hu)_{x})u+h_{x}u^{2}+2huu_{x}+ghh_{x}+hvu_{y}-fv=0
\]


\[
hu_{t}-h_{x}uu-huu{}_{x}+h_{x}u^{2}+2huu_{x}+ghh_{x}+hvu_{y}-fv=0
\]


\[
u_{t}+gh_{x}+uu_{x}+vu_{y}-fv=0
\]


Alltogether, we get the following non-conservative formulation:

\[
h_{t}+(hu)_{x}+(hv)_{y}=0
\]
\[
u_{t}+gh_{x}+uu_{x}+vu_{y}-fv=0
\]
\[
v_{t}+gh_{y}+uv_{x}+vv_{y}+fu=0
\]


We can also split these equations in linear and non-linear parts:

\[
L(U):=\left(\begin{array}{ccc}
\\
g\delta_{x} &  & -f\\
g\delta_{y} & f
\end{array}\right)U
\]


\[
N(U):=\left(\begin{array}{c}
(hu)_{x}+(hv)_{y}\\
uu_{x}+vu_{y}\\
uv_{x}+vv_{y}
\end{array}\right)
\]



\section{Vorticity formulation}

For certain reasons such as energy and enstropy conservation, the
equations are reformulated and split into the vorticity and H component.
We first add an artificial term $vv_{x}$

\[
u_{t}-vv_{x}+gh_{x}+uu_{x}+vv_{x}+vu_{y}-fv=0
\]


\[
u_{t}-vv_{x}+vu_{y}+(gh+u^{2}+v^{2})_{x}-fv=0
\]


and set $H:=gh+u^{2}+v^{2}$ and $\eta:=v_{x}-u_{y}+f$, yielding

\[
u_{t}-v\eta+H_{x}=0
\]


Similarly, we can reformulate $v_{t}$:

\[
v_{t}+gh_{y}+uv_{x}+vv_{y}+uu_{y}-uu_{y}+fu=0
\]


\[
v_{t}+uv_{x}-uu_{y}+fu+(gh+v^{2}+u^{2})=0
\]


This yields the vorticity formulation:

\[
h_{t}+(hu)_{x}+(hv)_{y}=0
\]


\[
u_{t}-v\eta+H_{x}=0
\]


\[
v_{t}+u\eta+H_{y}=0
\]



\section{Full linearization with perturbation}

The equations can be linearized around a perturbation $p=p_{0}+p'$,
with $p\in\{h,u,v\}$with $p_{0}$denoting a constant steady state.
Assuming $a'b'<O(\epsilon)$, we can use

$(ab)_{x}=((a_{0}+a')(b_{0}+b'))_{z}=(a'b_{0})_{z}+(b'a_{0})_{z}+(a'b')_{z}\approx(a'b_{0})_{z}+(b'a_{0})_{z}=b_{0}a'_{z}+a_{0}b'_{z}$ 

to simplify the non-conservative formulation to

\[
h'_{t}+h_{0}(u'_{x}+v'_{x})+h'_{x}(u_{0}+v_{0})=0
\]


Further, we assume a low velocity $\{|u_{0}|,|v_{0}|\}<O(\epsilon)$,
yielding

\[
h'_{t}+h_{0}u'_{x}+h_{0}v'_{x}=0
\]


Using the approximation $ab_{z}=(a_{0}+a')(b_{0}+b')_{z}=a_{0}b'_{z}+a'b'_{z}\approx a_{0}b'_{z}$
, we can then simplify the non-conservative formulation to

\[
u'_{t}+gh'_{x}+u_{0}u'_{x}+v_{0}u'_{y}-fv'=0
\]
\[
v'_{t}+gh'_{y}+u_{0}v'_{x}+v_{0}v'_{y}+fu'=0
\]


and again by assuming a low velocity, get the following formulation:

\[
h'_{t}+h_{0}u'_{x}+h_{0}v'_{x}=0
\]


\[
u'_{t}+gh'_{x}-fv'=0
\]
\[
v'_{t}+gh'_{y}+fu'=0
\]


Setting $U:=(h',u',v')$, we can write these equations as a linear
operator

\[
L(U):=\left(\begin{array}{ccc}
 & h_{0}\delta_{x} & h_{0}\delta_{y}\\
g\delta_{x} &  & -f\\
g\delta_{y} & f
\end{array}\right)U
\]


\[
U_{t}+L(U)=0
\]



\section{Linear/Non-linear version:}

In the equation above, we neglected also the non-linear velocities.
Assuming, that their influence is of high importance, we can write
the equations in a form which is not neglecting these equations:

\[
U_{t}+L(U)+N(U)=0
\]


with

\[
N(U):=\left(\begin{array}{c}
(hu)_{x}-h_{0}u_{x}+(hv)_{y}-h_{0}v_{y}\\
uu_{x}+vu_{y}\\
uv_{x}+vv_{y}
\end{array}\right)
\]


or with the assumption of negligible perturbations of the water height,
we get

\[
N(U):=\left(\begin{array}{c}
0\\
uu_{x}+vu_{y}\\
uv_{x}+vv_{y}
\end{array}\right)
\]

\end{document}
